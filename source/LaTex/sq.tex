\begin{document}
$
\vspace{5mm} f(x) = ax^{2} + bx + c, \hspace{2mm} a \neq 0 \\ \\
\vspace{5mm} x = \frac{-b \pm \sqrt{D}}{2a} \\
\vspace{5mm} D = b^{2} - 4ac \\
\vspace{5mm} f(x) = a(x - x_{1})(x - x_{2}) \\
\vspace{5mm} f(x) = a(x - h)^{2} + k \\
\vspace{5mm} ax^{2} + bx + c \\
\vspace{5mm} x = - \frac{b}{2a} \\
\vspace{5mm} y = - \frac{D}{4a} \\
\vspace{5mm} x_1 \hspace{5mm} x_2 \\
$
\end{document}